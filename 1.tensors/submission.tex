\documentclass[addpoints]{exam}
\pagestyle{headandfoot}
\usepackage{amsmath, amsfonts}
\usepackage{verbatim}
\usepackage{graphicx}
\usepackage[usenames,dvipsnames]{color}
\usepackage[utf8]{inputenc}
\usepackage{parskip}
\usepackage{hyperref}
\newcommand{\semester}{WS 2020/2021}
\runningheader{\students}{Submission}{\semester}
\runningfooter{}{\thepage}{}
\headrule
\footrule

% ---------- Modify team name, students, exercise number here ----------
\newcommand{\teamname}{shallow\_learning\_group}
\newcommand{\students}{Batuhan Karaca}
\newcommand{\assignmentnumber}{1}
% ---------- End Modify ----------

\title{Submission for Deep Learning Exercise \assignmentnumber}
\author{Team: \teamname\\Students: \students}
\date{\today}

\begin{document}
    \maketitle

    % ---------- Add Solution below here ----------

    \section*{Pen and Paper task: Probability Theory: Bertrand’s Box Paradox}
    Note that the variables $BB$, $WW$ and $BW$ denote the event for the card drawn, with two black sides, two white sides, and with one black and one white side respectively. Furthermore, $show B$ denotes when the drawn card's side facing up is black. Similarly white for $show W$.
    \subsection*{1)}
    \begin{align*}
        p(show B) &= p(show B|BB)p(BB) + p(show B|WW)p(WW) + p(show B|BW)p(BW)\\
        &= 1\frac{1}{3} + 0 + \frac{1}{2}\frac{1}{3}\\
        &= \frac{1}{2}\\
        p(show W) &= p(show W|BB)p(BB) + p(show W|WW)p(WW) + p(show W|BW)p(BW)\\
        &= 0 + 1\frac{1}{3} + \frac{1}{2}\frac{1}{3}\\
        &= \frac{1}{2}\\
    \end{align*}
    \subsection*{2)}
    \begin{align*}
        p(BB|show B) &= \frac{p(show B| BB)p(BB)}{p(show B)}\\
        &= \frac{1\frac{1}{3}}{\frac{1}{2}}\\
        &= \frac{2}{3}\\
    \end{align*}
    \subsection*{3)}
    \begin{align*}
        p(BW|show W) &= \frac{p(show W| BW)p(BW)}{p(show W)}\\
        &= \frac{\frac{1}{2}\frac{1}{3}}{\frac{1}{2}}\\
        &= \frac{1}{3}\\
    \end{align*}
    \section*{Question in BLR: How do the prior and posterior distributions differ? Why?}

    \begin{figure}
        \centering
        \includegraphics[width=\textwidth]{contour_lines.png}
        \caption{Contour lines of prior and post distributions}
        \label{fig:fig1}
    \end{figure}

    One can see the plot contours in figure~\ref{fig:fig1}. 
    It is easier to interpret a 2x2 covariance matrix $\Sigma$. $\Sigma_{11}$ corresponds to the variance of $\omega_0$, whereas $\Sigma_{22}$ to the variance of $\omega_1$. Furthermore, $\Sigma_{12}$ and $\Sigma_{21}$ are equal to the covariance between $\omega_0$ and $\omega_1$. Width of the contours are determined by the variance of $\omega_0$, whereas height of the contours are determined by the variance of $\omega_1$. The sign of the slope $\frac{d\omega_1}{d\omega_0}$ of the semi-major axis (see \href{https://en.wikipedia.org/wiki/Ellipse}{here}) of the elliptic contours is the sign of the covariance (or linear dependence) between $\omega_0$ and $\omega_1$. The prior distribution is $N(\textbf{0},\textbf{I})$, hence symmetrical along all axes with zero mean. However, the fitted weights are negatively correlated (slope is negative) with smaller variances (less uncertanty); hence, we can say that the models estimations became more correlated, and less random.
\end{document}
